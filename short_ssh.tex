\documentclass[11pt,spanish]{article}

\usepackage[a6paper, landscape,
	left=1cm, top=1cm, right=1cm, bottom=1.5cm
]{geometry}
%\usepackage[a4paper, margin=3cm]{geometry}
\usepackage{babel}
\usepackage{xltxtra}
\usepackage{grid-system}
\usepackage{multicol} % simpler than grid for lots of cases
\usepackage{hyperref}
\usepackage{xcolor}
\usepackage{graphicx}
\usepackage[export]{adjustbox}
\usepackage{fancyhdr}
\usepackage{listings}

\setmainfont{Comic Sans MS}
%\setmathfont{eulervm}
%\lstset{basicstyle=\footnotesize\ttfamily}
\lstset{basicstyle=\footnotesize\setmainfont{DejaVu Sans Mono}}

\definecolor{verletblue}{HTML}{487AD8}
\definecolor{verletgreen}{HTML}{37DE88}
\definecolor{verletcyan}{HTML}{1AFBFF}

\fancyhead{}
\fancyfoot{}
\fancyfoot[RO]{{\footnotesize \thepage\ de\ \pageref*{lastpage}}}
\renewcommand{\headrule}{}
%\renewcommand{\footrule}{\vbox to 0pt{\hbox to\headwidth{\dotfill}\vss}}
% XXX should find way to use only one right aligned rule...
\renewcommand{\footrule}{{\footnotesize \rule{\textwidth - 5ex}{0pt}\rule{5ex}{1pt}\vss}}
\pagestyle{fancy}

\newcommand{\fr}[1]{%
	\begin{flushright}
		#1
	\end{flushright}
}
%row space
\newcommand{\rowsp}[1][1em]{\vspace{#1}}
\newcommand{\hone}[1]{{\rowsp[0.3em]\noindent\Large #1 \rowsp[0.3em]}}
\newcommand{\htwo}[1]{{\rowsp\noindent\large #1 \rowsp}}
\newcommand{\htworuler}[1]{{%
	\rowsp\noindent\Large #1%
	\\ {\color{verletgreen}\noindent\rule{\textwidth + 2em}{0.5em}}\rowsp%
}}
\newcommand{\hthree}[1]{{\rowsp\noindent\large #1 \rowsp[0.5em]}}
\newcommand{\hfour}[1]{{\rowsp\large #1 \rowsp[0.5em]}}
\newcommand{\for}[1]{{#1 \rowsp}}
\newcommand{\signline}{\rule{\textwidth}{1pt}}
\newcommand{\emptycell}[1][1]{\begin{Cell}{#1}\ \end{Cell}}

%=======================
% specific to presentations
%\newcommand{\myitm}{aoeuaeou}
\newcommand{\myitm}[1]{\begin{itemize}#1\end{itemize}}

\newcommand{\mydesc}[1]{%
	\begin{description}
	\setlength\itemsep{0em}%
	#1
	\end{description}
}
\newcommand{\pros}{\item[pros:]}
\newcommand{\cons}{\item[cons:]}

\setlength{\parindent}{0pt}
\setlength{\parskip}{1ex plus 0.5ex minus 0.2ex}
%=======================

%\renewcommand{\emph}[1]{\emph{#1}}

\title{Nombres cortos en ssh(1)}
\author{Abel Camarillo $<$acamari@verlet.org$>$}


\begin{document}

\maketitle
\thispagestyle{empty}


\newpage %===============
\hone{Supuestos}

Hay que conectarse a varios clientes:
\myitm{
	\item neuroservices
	\item linkaform
	\item movivendor
	\item cloudsigma
}

\newpage %===============

\hone{Supuestos (cont.)}

Cada cliente usa varios dominios:
\begin{multicols}{2}
\myitm{
	\item Neuroservices
	\myitm{
		\item neuroservices.com.mx
		\item neuroservices.net.mx
		\item neuroservices.mx
		\item e-guard.com.mx
		\item e-guard.net.mx
		\item e-guard.mx
	}

	\item Linkaform
	\myitm{
		\item lkf.cloud
		\item linkaform.com.mx
		\item linkaform.mx
	}
	\columnbreak

	\item Movivendor
	\myitm{
		\item movivendor-old.com
		\item movivendor-new.mx
		\item movivendor.com.mx
	}

}
\end{multicols}

\newpage %===============

\hone{Opciones}

\myitm{
	\item \$HOME/.ssh/config
	\item /etc/hosts
	\item /etc/resolv.conf (y derivados)
}

\newpage %===============

\hone{\$HOME/.ssh/config}

\begin{lstlisting}
	$ cat $HOME/.ssh/config
	Host www-neuroservices
	HostName www.neuroservices.net.mx

	Host www-linkaform
	HostName www.linkaform.com.mx
\end{lstlisting}

\myitm{
	\pros No necesita root
	\cons Sólo funciona con ssh, no con más programas
	\cons Más administración: dar de alta, ¿manejo de actualizaciones?
	\cons ¿Cómo distinguir el dominio \lstinline{www} entre clientes?
	\myitm{
		\item Usar sufijos
		\item Ya no está realmente corto...
	}
}

\newpage %===============

\hone{/etc/hosts}

\begin{lstlisting}
	$ cat > /etc/hosts
	1.1.1.1 www-neuroservices
	2.2.2.2 www-linkaform
\end{lstlisting}

\myitm{
	\cons Mismos que \$HOME/.ssh/config
	\cons Más administración:
	\myitm{
		\item Sólo permite dominio $->$ ip
		\item No permite dominio $->$ dominio
	}
	\pros Funciona con todos los programas: ssh(1), ping(8), nc(1), curl(1), etc.
}

\newpage %===============

\hone{/etc/resolv.conf}

\begin{lstlisting}
	$ cat > /etc/hosts
	search neuroservices.com.mx neuroservices.net.mx \
		neuroservices.mx
	^D
	$ ssh x;
\end{lstlisting}

Busca x en x.neuroservices.com.mx, sino en x.neuroservices.net.mx sino en
x.neuroservices.mx.

\myitm{
	\cons Requiere root
	\pros Funciona con todos los programas
	\pros Baja administración
	\pros Unificación con los nombres globales (no genera nombres nuevos)
	\item ¿Cómo cambiar de cliente?
}

\newpage %===============

\hone{LOCALDOMAIN}

\begin{lstlisting}
	$ export LOCALDOMAIN="neuroservices.com.mx \
		neuroservices.net.mx \
		neuroservices.mx"
	$ ssh www; # www.neuroservices.com.mx

	$ export LOCALDOMAIN="linkaform.com.mx
		lkf.cloud \
		linkaform.mx"
	$ ssh postgres; # postgres.lkf.cloud
\end{lstlisting}

\myitm{
	\pros Mismos que el anterior
	\pros No requiere root
	\pros Fácil cambiar de cliente
	\pros Integrable a .bashrc / .kshrc
}

\newpage %===============

\hone{LOCALDOMAIN}

Integrable con la shell, fácil cambiar de usuario:

\begin{lstlisting}
	$ cat >> ~/.kshrc
	alias neuro='export LOCALDOMAIN="neuroservices.com.mx \
		neuroservices.net.mx \
		neuroservices.mx"'

	alias linkaform='export LOCALDOMAIN="linkaform.com.mx \
		lkf.cloud \
		linkaform.mx"'

	^D
	$ ksh -l # reloads .kshrc
	$ ssh postgres; # NOT FOUND
	$ neuro
	$ ssh postgres; # NOT FOUND
	$ linkaform
	$ ssh postgres; # postgres.lkf.cloud
\end{lstlisting}

\newpage %===============
\ 
\vspace{\stretch{1}}
\begin{center}
\hone{Gracias}
\end{center}
\vspace{\stretch{1}}

\label{lastpage}
\end{document}
